\documentclass[12pt]{scrartcl}

\usepackage{fullpage}

\setlength{\parindent}{0pt}
\setlength{\parskip}{.25cm}

\usepackage{graphicx}

\usepackage{xcolor}

\definecolor{darkred}{rgb}{0.5,0,0}
\definecolor{darkgreen}{rgb}{0,0.5,0}
\usepackage{hyperref}
\hypersetup{
  letterpaper,
  colorlinks,
  linkcolor=red,
  citecolor=darkgreen,
  menucolor=darkred,
  urlcolor=blue,
  pdfpagemode=none,
  pdftitle={CS2 - Lab 2.0 - Worksheet},
  pdfkeywords={}
}

\definecolor{MyDarkBlue}{rgb}{0,0.08,0.45}
\definecolor{MyDarkRed}{rgb}{0.45,0.08,0}
\definecolor{MyDarkGreen}{rgb}{0.08,0.45,0.08}

\definecolor{mintedBackground}{rgb}{0.95,0.95,0.95}
\definecolor{mintedInlineBackground}{rgb}{.90,.90,1}

%\usepackage{newfloat}
\usepackage[newfloat=true]{minted}
\setminted{mathescape,
               linenos,
               autogobble,
               frame=none,
               framesep=2mm,
               framerule=0.4pt,
               %label=foo,
               xleftmargin=2em,
               xrightmargin=0em,
               startinline=true,  %PHP only, allow it to omit the PHP Tags *** with this option, variables using dollar sign in comments are treated as latex math
               numbersep=10pt, %gap between line numbers and start of line
               style=default, %syntax highlighting style, default is "default"
               			    %gallery: http://help.farbox.com/pygments.html
			    	    %list available: pygmentize -L styles
               bgcolor=mintedBackground} %prevents breaking across pages
               
\setmintedinline{bgcolor={mintedBackground}}
\setminted[text]{bgcolor={mintedBackground},linenos=false,autogobble,xleftmargin=1em}
%\setminted[php]{bgcolor=mintedBackgroundPHP} %startinline=True}
\SetupFloatingEnvironment{listing}{name=Code Sample}
\SetupFloatingEnvironment{listing}{listname=List of Code Samples}

\begin{document}

\section*{CSCE 156 - Lab 2.0 - Conditionals \& Loops - Worksheet}

Names: \underline{\hspace{10cm}}

If you've had prior Java experience you should complete the 
``PHP Introduction'' Lab.  If you have not had prior Java 
experience or feel that you are deficient in Java, you should 
complete the ``Java Introduction'' Lab.  It is not required to 
complete both, but you are encouraged to do so.

\begin{enumerate}
  \item Complete the first activity and demonstrate it for values of $n$
  including 10, 25, and 100.  Write the answers below and demonstrate 
  it to a lab instructor.

  \item Complete the second activity and demonstrate a working version 
  to a lab instructor.  Answer the following:
  \begin{enumerate}
    \item How much of a credit do you get for the ``Tom, Dick, and Harry'' children? 
    \item Modify the children to have ages 2, 18, and 19 respectively.  
    How much of a credit are they worth?
  \end{enumerate}
\end{enumerate}

Lab Instructor Signature\underline{\hspace{7.5cm}}

\end{document}
